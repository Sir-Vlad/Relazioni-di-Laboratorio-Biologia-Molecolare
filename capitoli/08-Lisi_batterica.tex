\section{Preparazione dei componenti per la procedura \ref*{sec:espProteine}}

\subsection{Materiali e composti utilizzati}
\begingroup
\sisetup{
	propagate-math-font = true ,
	reset-math-version = false
}
\paragraph{Strumentazione}
\begin{itemize}
	\itemb[Beuta da {\boldmath \qty{500}{\ml}}]
	\itemb[Ansa]: è uno strumento utilizzato per prelevare e inoculare piccole quantità di microbi o colture microbiche.
	\itemb[Tubi da batteriologia]: contenitori utilizzati per la coltura e la crescita di microrganismi. Sono realizzati in vetro o plastica resistente agli agenti chimici e autoclavabili per garantire la sterilità.
\end{itemize}

\paragraph{Composti utilizzati}
\begin{itemize}\label{it:compostiUtilizzatiRel8}
	\itemb[\iupac{Triptone}]: Descritto a pagina \pageref{it:prepLBsolido-CompostiUtilizzati}
	\itemb[Estratto di lievito]: Descritto a pagina \pageref{it:prepLBsolido-CompostiUtilizzati}
	\itemb[Cloruro di sodio (\ch{NaCl})]: Descritto a pagina \pageref{it:prepLBsolido-CompostiUtilizzati}
	\itemb[Saccarosio]: ha una funzione di stabilizzazione per le proteine.
	\itemb[\gls{tris} (\pH\ 8)]: utilizzato per preparare soluzioni tampone a diversi valori di pH, fornendo una condizione stabile per reazioni enzimatiche, colture cellulari, elettroforesi su gel e altre applicazioni che richiedono un controllo rigoroso del pH
	\itemb[\gls{edta}]: è un agente chelante, il che significa che forma complessi stabili con ioni metallici, in particolare con ioni di metalli bivalenti
	\itemb[Triton X-100]: Il Triton è un detergente che va a legare i lipidi.
	\itemb[\iupac{\b-mercaptoetanolo}] è un agente riducente capace di rompere di ponti disolfuro delle cisteine.
	\itemb[\gls{tris-hcl} (\pH\ 7.5)]: è un tampone chimico ampiamente utilizzato in laboratori scientifici per regolare e mantenere il \pH\ delle soluzioni. È composto da due componenti principali: il \gls{tris} e l'acido cloridrico (\ch{HCl})
	\itemb[Ampicillina]: Descritta a pagina \pageref{it:prepLBsolido-CompostiUtilizzati}.
\end{itemize}

\subsection{Protocollo}
Le seguenti operazioni servono come preparazione alla procedura \ref{sec:espProteine}.
\subsubsection{Sterilizzazione delle beute da {\boldmath \qty{500}{\ml}}}
La coltura di \enzima{Escherichia coli} verrà fatta crescere in una beuta di Pirex. Per essere sicuri che crescerà solo l'\enzima{Escherichia coli} bisogna sterilizzare la beuta nel seguente modo:
\begin{enumerate}
	\item Tagliare un foglio di alluminio di grandezza sufficiente a coprire l’apertura della beuta.
	\item Prendere una beuta di vetro Pirex da \qty{500}{\ml} e coprirla l'apertura con l’alluminio.
	\item Attaccare al foglio di alluminio un pezzo di scotch da autoclave
	\item Sterilizzare la beuta in autoclave a \qty{120}{\celsius} per \qty{20}{\min}
\end{enumerate}

\endgroup

\subsubsection{Preparazione del terreno LB}
\begin{enumerate}
	\item Pesare \qty{10}{\g} di \iupac{triptone} in una vaschetta da laboratorio e versarlo in un becher.
	\item Pesare \qty{5}{\g} di estratto di lievito in una vaschetta da laboratorio e versarlo nel becher.
	\item Pesare \qty{5}{\g} di cloruro di sodio in una vaschetta da laboratorio e versarlo nel becher.
	\item Aggiungere \qty{0.9}{\l} di acqua distillata e mescolare fino alla solubilizzazione delle polveri.
	\item Versare la soluzione in un cilindro e aggiungere acqua distillata fino a \qty{1}{\litre} di volume.
	\item Trasferire in una bottiglia e chiuderla.
	\item Mettere la bottiglia in autoclave a \qty{120}{\celsius} per \qty{20}{\min}
\end{enumerate}
Il terreno risultante non dovrà presentare un aspetto torbido (\autoref{img:8-01}).

\subsubsection{Preparazione del tampone dei lisi batterica}
\noindent Preparazione di \qty{25}{\ml} di tampone di lisi (\autoref{img:8-02}) avente composizione:
\begin{itemize}[person]
	\item \qty{50}{\milli\Molar} di Tris a \pH\ 8
	\item \qty{25}{\percent} m/V di saccarosio
	\item \qty{1}{\milli\Molar} di EDTA
\end{itemize}

\paragraph{Procedimento}
\begin{enumerate}
	\item Calcolare la massa da pesare del saccarosio tramite la proporzione \(25:100=x:25\), ottenendo \qty{6.25}{\g} da pesare, i quali vengono inseriti in un tubo Falcon da \qty{50}{\ml}.
	\item Aggiungere acqua distillata fino a raggiungere i \(3/4\) dei \qty{25}{\ml} del volume finale.
	\item Chiudere la Falcon e agitare fino allo scioglimento del saccarosio.
	\item Calcolare il volume da prelevare di Tris ed EDTA, usando la formula di diluizione \mbox{\(V_1\cdot C_1 = V_2\cdot C_2\)}:
	      \begin{itemize}[person]
		      \item Tris da stock \qty{1}{\Molar} si ottiene \qty{1.25}{\ml}
		      \item EDTA da stock \qty{0.5}{\Molar} si ottiene \qty{50}{\micro\litre}
	      \end{itemize}
\end{enumerate}

\subsubsection{Preparazione del tampone Triton}
\noindent Preparazione di \qty{25}{\ml} di tampone Triton (\autoref{img:8-02}) avente composizione:
\begin{itemize}[person]
	\item \qty{0.5}{\percent} m/V di Triton X-100
	\item \qty{20}{\milli\Molar} Tris-\ch{HCl} \pH\ \num{7.5}
	\item \qty{1}{\milli\Molar} di \iupac{\b-mercaptoetanolo}
\end{itemize}

\paragraph{Procedimento}
\begin{enumerate}
	\item Calcolare la massa da pesare di Triton tramite la proposizione \(25:100=x:0.5\), ottenendo \qty{0.125}{\g}, i quali vengono inseriti in un tubo Falcon da \qty{50}{\ml} usando una pipetta monouso che ha un collo più grande a causa della viscosità del liquido
	\item Aggiungere acqua distillata fino a raggiungere i \(3/4\) dei \qty{25}{\ml} di volume finale.
	\item Calcolare il volume da prelevare di \iupac{\b-mercaptoetanolo} e di Tris-\ch{HCl}, usando la formula di diluizione \mbox{\(V_1\cdot C_1 = V_2\cdot C_2\)}:
	      \begin{itemize}
		      \item \iupac{\b-mercaptoetanolo} da stock \qty{1}{\milli\Molar} si ottiene \qty{2.5}{\micro\litre}
		      \item Tris-\ch{HCl} da stock \qty{1}{\milli\Molar} si ottiene \qty{0.5}{\ml}
	      \end{itemize}
	\item Aggiungere acqua distillata fino a raggiungere \qty{25}{\ml} di volume.
\end{enumerate}
Il tampone Triton viene utilizzato per pulire i corpi di inclusione.


\subsubsection{Preparazione del preinoculo BL21\texorpdfstring{\,--\,}{--}LHCSR }
\begin{Attenzione}
	Durante la preparazione del preinoculo, lavorare sempre sotto cappa biologica e tutto il materiale utilizzato deve essere sterile per non inquinare la preparazione
\end{Attenzione}
\begin{enumerate}
	\item Trasferire \qty{3}{\ml} di terreno LB sterile in una provetta da preinoculo.
	\item Aggiungere \qty{3}{\micro\litre} di ampicillina 1000X.
	      \begin{Note}
		      L'aggiunta di ampicillina serve per non far rilasciare il plasmide inserito dai batteri perché rende necessaria l'espressione dei geni del plasmide per la resistenza all'antibiotico.
	      \end{Note}
	\item Prelevare una singola colonia dalla piastra con un’ansa sterile e trasferirla nel terreno liquido.
	\item Incubare in agitazione a \qty{37}{\celsius}.
	      \begin{Note}
		      L'agitazione dell'incubazione serve per l'ossigenazione del terreno che favorisce la crescita batterica. Questo è permesso dai tubi da batteriologia che hanno due step di chiusura della provetta. Il primo step permette l'areazione della provetta mentre il secondo step chiude la provetta completamente.
	      \end{Note}
\end{enumerate}

\photo{
	\img{8-01}{\\\slz di terreno LB}{8-01}{}
}{
	\img{8-02}{\\Tampone Triton e di lisi batterica}{8-02}{}
}{
	\img{8-03}{\\Preinoculo BL21\,--\,\glsxtrshort{lhcsr}}{8-03}{}
}