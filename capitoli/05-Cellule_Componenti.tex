\section{Preparazione di terreno LB solido}
\subsection{Materiali e composti utilizzati}
\paragraph{Strumentazione}
\begin{itemize}
	\itemb[Autoclave:] è un dispositivo utilizzato per svolgere il processo di sterilizzazione delle attrezzature (\autoref{img:5-02}), mediante vapore saturo a \qty{121}{\celsius} a \qty{1}{\bar}, per una durata che varia a seconda della dimensione del carico e del suo contenuto.
	
	La temperatura a cui opera è stata scelta per eliminare la maggior parte dei microorganismi comuni. Il tempo di attesa eccede i \qty{20}{\min} a causa delle fasi di riscaldamento e raffreddamento dell’autoclave.
\end{itemize}

\paragraph{Composti utilizzati}
\begin{itemize}
	\item
\end{itemize}

\subsection{Protocollo}
\subsection{Preparazione del terreno LB solido}
\begin{enumerate}
	\item Inserire nella bottiglia:
	\begin{itemize}
		\item \qty{0.5}{\g} di triptone
		\item \qty{0.25}{\g} di estratto di lievito
		\item \qty{0.25}{\g} di \ch{NaCl}
		\item \qty{0.75}{\g} di agar batteriologico
		\item \qty{50}{\ml} di acqua distillata
	\end{itemize}
	\begin{myBox}{NOTA}
		Sulla bottiglia attaccare un pezzo di nastro adesivo da autoclave. Questo tipo di nastro ti indica se la sterilizzazione è avvenuta o meno a seconda se si è colorato di nero.
	\end{myBox}
	\item Sterilizzare la bottiglia in autoclave a \qty{120}{\celsius}  per \qty{20}{\min}. 
	
	\textbf{ATTENZIONE:} Il tappo della bottiglia deve essere allentato per evitare che esploda. 
	\item Aspettare che la soluzione si raffreddi.
	\item Dividere la soluzione autoclavata in due da 25 mL (una in una falcon sterile) per la preparazione delle piastre:
	\begin{enumerate}[label=\Alph*)]
		\item Piastra con \ac{KAN} e \ac{CAF}
		\begin{enumerate}[label=\arabic*.]
			\item Aggiungere \qty{25}{\micro\litre} di \acs{CAF} 1000X .			
			\item Aggiungere \qty{25}{\micro\litre} di \acs{KAN} 1000X.			
			\item Mescolare bene.			
			\item Versare in una piastra petri.
		\end{enumerate}
		\item Piastra con \ac{AMP}, \acs{IPTG}, \acs{X-gal}. L’IPTG induce l’espressione della \iupac{\b-galattosidasi}, mentre l’\acs{X-gal} si colora di blu in presenza di questo enzima.
		\begin{enumerate}[label=\arabic*.]
			\item Aggiungere \qty{100}{\ug\per\ml} di \acl{AMP} 1000X (\qty{25}{\micro\litre}).			
			\item Aggiungere \qty{80}{\ug\per\ml} di \acs{X-gal} (\qty{25}{\micro\litre}).			
			\item Aggiungere \qty{0.5}{\milli\Molar} \acs{IPTG} (\qty{25}{\micro\litre}).			
			\item Mescolare 			
			\item Versare il terreno in una piastra
		\end{enumerate}
	\end{enumerate}
	\item Lasciare le due piastre semiaperte sotto la cappa biologica fino a quando l’agar non si sarà solidificato.
	\item Chiuderle e conservarle a \qty{4}{\celsius}.
\end{enumerate}

\subsection{Preparazione delle cellule competenti}
\begin{enumerate}
	\item Sotto cappa biologica, prelevare con pipette sierologiche:
	\begin{itemize}
		\item \qty{5}{\ml} di una coltura di cellule DH5\textalpha
		\item \qty{5}{\ml} di BL21 (\(\text{OD}_{600}\) tra \numrange{0.3}{0.4}) 
	\end{itemize}
	trasferire in 2 provette Falcon sterili da \qty{50}{\ml} 
	\item Centrifugare \qty{5}{\min} a 3000 g a \qty{4}{\celsius}
	\item Eliminare il surnatante
	\item Risospendere le cellule in \qty{0.8}{\ml} di buffer 1.
	\item Trasferire sotto cappa le due soluzioni in eppendorf da \qty{2}{\ml}.
	\item Incubare in ghiaccio \qty{15}{\min}.
	\item Centrifugare \qty{5}{\min} a 3000 g a \qty{4}{\celsius}
	\item Eliminare sotto cappa il surnatante.
	\item Risospendere le cellule in \qty{0.4}{\ml} di buffer 2.
	\item Congelare in azoto liquido.
	\item Conservare a \qty{-80}{\celsius}.
\end{enumerate}
Le cellule competenti sono trattate con \ch{CaCl2} che altera la parete cellulare dei batteri rendendola attrattiva per il DNA.




\begingroup
	\newpage
	\photo{
		\img{5-01}{}{5-01}{}
	}{
		\img{5-02}{\\Termociclatore}{5-02}{}
	}{
		\img{5-03}{}{5-03}{}		
	}

	\vspace*{0.5cm}
	\photo{
		\img{5-04}{}{5-04}{}
	}{
		\img{5-05}{}{5-05}{}
	}{
		\img{5-06}{}{5-06}{}		
	}
\endgroup