\section{Estrazione dell'RNA con \trizol}
\subsection{Materiali e composti utilizzati}
\paragraph{Strumentazione}
\begin{itemize}
	\item Provetta \foreignlanguage{german}{Eppendorf}
	\item Mortaio e pestello
	\item Propipette
	\item \foreignlanguage{english}{Vortex}
	\item Freezer
\end{itemize}

\paragraph{Composti utilizzati}
\begin{itemize}
	\item Foglie 
	\item Azoto liquido: liquefazione dell'azoto gassoso a temperatura di \qty{-196}{\celsius}. Lo utilizziamo per facilitare la rottura delle pareti cellulari vegetali
	\item \trizol: \slz di fenolo e tiocianato di gauanidinio, che ha il compito di denaturare le cellule e dissolvere i componenti cellulari ma di mantenere integro il DNA 
	\item Cloroformio
	\item Isopropanolo
	\item Etanolo \qty{70}{\percent}
	\item Acqua distillata
\end{itemize}

\subsection{Protocollo}
\begin{enumerate}
	\item La lisi di una cellula vegetale non può avvenire esclusivamente attraverso l’utilizzo di agenti chimici, a causa della presenza della parete cellulare. Triturare le foglie in un mortaio (lisi meccanica) versando azoto liquido fino a ottenere una polvere. La bassa temperatura impedisce agli enzimi di degradare il DNA.
	\item Trasferire circa \qty{100}{\mg} di campione (1-2 spatole) all’interno di una provetta \foreignlanguage{german}{Eppendorf} da \qty{2}{\ml}.
	\item Sospendere i tessuti triturati in \trizol (\qty{1}{ml} per \qtyrange{50}{100}{mg} di tessuto) e vortexare per \qty{10}{\sec}.
	\item Lasciar riposare il composto a temperatura ambiente per \qty{5}{\min} per assicurare la completa dissociazione dei complessi nucleoproteici.
	\item Aggiungere \qty{0.2}{\ml} di cloroformio per ml di \trizol usati.
	\item Vortexare per \qty{15}{\sec} e lasciar riposare per \qty{15}{\min} a temperatura ambiente.
	\item Centrifuga per \qty{15}{\min} a \qty{4}{\celsius}. La centrifugazione permette di ottenere 3 fasi: 
	\begin{itemize}
		\item una fase organica rosso scuro/marrone contenente proteine e lipidi
		\item un’interfase contenente DNA genomico precipitato
		\item una fase acquosa superiore contenente l’RNA
	\end{itemize}
	\item Separare la fase acquosa da quella organica nella cappa chimica.
	\item Aggiungere \qty{0.5}{ml} d'isopropanolo per \unit{\ml} di \trizol usati e mescolare.
	\item Lasciar riposare a in freezer a \qty{-20}{\celsius} per \qty{20}{\min}
	\item Centrifugare per \qty{10}{\min} a \qty{4}{\celsius}. L’RNA precipita.
	\item Rimuovere la soluzione acquosa sotto la cappa chimica e aggiungere \qty{1}{ml} di etanolo \qty{70}{\percent} per \unit{\ml} di \trizol usato.
	\item Vortexare il campione e centrifugare per \qty{5}{\min} a \qty{4}{\celsius}.
	\item Svuotare la fase liquida e far asciugare all’aria per \qty{10}{\min}
	\item Risospendere il composto \qtyrange{50}{100}{\micro\litre} di acqua distillata.
	\item Vortexare	
\end{enumerate}


\begingroup
\newpage
\photo{
	\img{2-01.jpeg}{\\Foglie macinate}{img:2-01}{trim={15cm 25cm 15cm 15cm},clip}
}{
	\img{2-02.jpeg}{\\Aggiunta di \trizol}{img:2-02}{trim={15cm 30cm 15cm 10cm},clip}
}{
	\img{2-03.jpeg}{\\Soluzione dopo vortex}{img:2-03}{trim={5cm 15cm 5cm 15cm},clip}
}

\vspace{0.5cm}
\photo{
	\img{2-04.jpeg}{\\Aggiunta cloroformio}{img:2-04}{trim={10cm 20cm 10cm 20cm},clip}
}{
	\img{2-05.jpeg}{\\Centrifugazione soluzione}{img:2-05}{trim={10cm 20cm 10cm 20cm},clip}
}{
	\img{2-06.jpeg}{\\Recupero della fase acquosa}{img:2-06}{trim={10cm 30cm 10cm 10cm},clip}
}

\vspace{0.5cm}
\photo{
	\img{2-07.jpeg}{\\Aggiunta del \iupac{2-propanolo}}{img:2-07}{trim={5cm 20cm 15cm 20cm},clip}
}{
	\img{2-08.jpeg}{\\RNA precipitato}{img:2-08}{trim={5cm 20cm 15cm 20cm},clip}
}{
	\img{2-09.jpeg}{\\Aggiunta di etanolo \qty{70}{\percent}}{img:2-09}{trim={7cm 20cm 13cm 20cm},clip}
}
\endgroup