\section{Estrazione dell'RNA con \trizol}
\subsection{Materiali e composti utilizzati}
\paragraph{Strumentazione}
\begin{itemize}[person]
	\item Provetta \foreignlanguage{german}{Eppendorf}
	\item Mortaio e pestello
	\item Propipette
	\item \foreignlanguage{english}{Vortex}
	\item Freezer
\end{itemize}

\paragraph{Composti utilizzati}

\begingroup
\sisetup{
propagate-math-font = true ,
reset-math-version = false
}
\begin{itemize}[person]
	\itemb[Foglie verdi] 
	\itemb[Azoto liquido]: liquefazione dell'azoto gassoso a temperatura di \qty{-196}{\celsius}. Lo utilizziamo per facilitare la rottura delle pareti cellulari vegetali
	\itemb[\trizol]: \slz di fenolo e tiocianato di gauanidinio, che ha il compito di denaturare le cellule e dissolvere i componenti cellulari ma di mantenere integro il DNA 
	\itemb[Cloroformio]: ha il compito di separare le componenti proteiche e il DNA dalla nostra soluzione
	\itemb[Isopropanolo]: ha il compito di far precipitare l'RNA
	\itemb[Etanolo {\boldmath \qty{70}{\percent}}]
	\itemb[Acqua distillata]
\end{itemize}
\endgroup

\subsection{Protocollo}
\begin{enumerate}
	\item Triturare le foglie in un mortaio (\textit{lisi meccanica}) versando azoto liquido fino a ottenere una polvere.
	\begin{Note}
		La lisi di una cellula vegetale non può avvenire esclusivamente attraverso l’utilizzo di agenti chimici, a causa della presenza della parete cellulare.  
		
		Inoltre, la bassa temperatura impedisce agli enzimi di degradare il DNA.
	\end{Note}
	\item Trasferire circa \qty{100}{\mg} di campione (\numrange{1}{2} spatole) all’interno di una provetta \foreignlanguage{german}{Eppendorf} da \qty{2}{\ml}.
	\item Sospendere i tessuti triturati in \trizol (\qty{1}{ml} per \qtyrange{50}{100}{\mg} di tessuto) e vortexare per \qty{10}{\sec}.
	\item Lasciar riposare il composto a temperatura ambiente per \qty{5}{\min} per assicurare la completa dissociazione dei complessi nucleoproteici.
	\item Aggiungere \qty{0.2}{\ml} di cloroformio per ml di \trizol usati.
	\item Vortexare per \qty{15}{\sec} e lasciar riposare per \qty{15}{\min} a temperatura ambiente.
	\item Centrifugare a \qty{12000}{\giri} per \qty{15}{\min} a \qty{4}{\celsius}. La centrifugazione permette di ottenere 3 fasi: 
	\begin{itemize}
		\item una \emph{fase organica} di colore rosso scuro/marrone contenente proteine e lipidi
		\item un’\emph{interfase} contenente DNA genomico precipitato
		\item una \emph{fase acquosa} superiore contenente l’RNA
	\end{itemize}
	\item Separare la fase acquosa da quella organica sotto cappa chimica, trasferendola in una nuova provetta.
	\item Aggiungere \qty{0.5}{ml} di isopropanolo per \unit{\ml} di \trizol usati e mescolare.
	\item Lasciar riposare a in freezer a \qty{-20}{\celsius} per \qty{20}{\min}
	\item Centrifugare \qty{12000}{\giri} per \qty{10}{\min} a \qty{4}{\celsius}. Dopo la centrifugazione l’RNA sarà presente come corpo di fondo.
	\item Rimuovere la fase acquosa sotto cappa chimica e aggiungere \qty{1}{ml} di etanolo \qty{70}{\percent} per \unit{\ml} di \trizol usato.
	\item Vortexare il campione e centrifugare per \qty{5}{\min} a \qty{4}{\celsius}.
	\item Svuotare la fase liquida e far asciugare all’aria per \qty{10}{\min}
	\item Risospendere il composto in \qtyrange{50}{100}{\micro\litre} di acqua distillata e vortexare	
\end{enumerate}
Ora la soluzione ottenuta si può conservare per usi futuri.

\subsection{Quantificazione del DNA e RNA mediante spettroscopia UV}
% ADD: spiegare perchè non usate le cuvette FORSE
\noindent Si misurerà la concentrazione del pUC18 (\autoref{sec:miniprep})
\begin{enumerate}
	\item Depositare una goccia (\qty{2}{\micro\litre}) di acqua sul tip del nanodrop
	\item Abbassare delicatamente il braccio e registrare il bianco
	\item Sollevare il braccio, asciugare il tip con della carta assorbente
	\item Depositare una goccia (\qty{2}{\micro\litre}) di campione sul tip del nanodrop
	\item Abbassare delicatamente il braccio e registrare lo spettro
	\item Calcolare la quantità di RNA (\(A=1\to\) \qty{40}{\ng\per\micro\litre}) e la sua purezza (rapporto \(A_{260}/A_{280}\))
\end{enumerate}
Congelare l'RNA non diluito e il plasmide pUC18.





\begingroup
\newpage
\photo{
	\img{2-01.jpeg}{\\Foglie macinate}{2-01}{trim={15cm 25cm 15cm 15cm},clip}
}{
	\img{2-02.jpeg}{\\Aggiunta di \trizol}{2-02}{trim={15cm 30cm 15cm 10cm},clip}
}{
	\img{2-03.jpeg}{\\Soluzione dopo vortex}{2-03}{trim={5cm 15cm 5cm 15cm},clip}
}

\vspace{0.5cm}
\photo{
	\img{2-04.jpeg}{\\Aggiunta cloroformio}{2-04}{trim={10cm 20cm 10cm 20cm},clip}
}{
	\img{2-05.jpeg}{\\Centrifugazione soluzione}{2-05}{trim={10cm 20cm 10cm 20cm},clip}
}{
	\img{2-06.jpeg}{\\Recupero della fase acquosa}{2-06}{trim={10cm 30cm 10cm 10cm},clip}
}

\vspace{0.5cm}
\photo{
	\img{2-07.jpeg}{\\Aggiunta del \iupac{2-propanolo}}{2-07}{trim={5cm 20cm 15cm 20cm},clip}
}{
	\img{2-08.jpeg}{\\RNA precipitato}{2-08}{trim={5cm 20cm 15cm 20cm},clip}
}{
	\img{2-09.jpeg}{\\Aggiunta di etanolo \qty{70}{\percent}}{2-09}{trim={7cm 20cm 13cm 20cm},clip}
}
\endgroup