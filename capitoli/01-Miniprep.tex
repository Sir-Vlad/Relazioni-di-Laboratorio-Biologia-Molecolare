\section{Miniprepazione del DNA plasmidico}

\subsection{Materiali e composti utilizzati}
\paragraph{Strumentazione}
\begin{itemize}[person]
	\item Pipette \foreignlanguage{german}{Eppendorf} da \qtylist{1.5;2.5}{\ml}
	\item Propipette
	\item \foreignlanguage{english}{Vortex}
	\item Centrifuga
	\item Freezer
\end{itemize}

\paragraph{Composti utilizzati}
\begin{itemize}[person]
	\item \textbf{Coltura batterica} di batteri resistenti alla amoxicillina (antibiotico)
	\item \textbf{\slz I}: soluzione contenente EDTA, composto chelante che ha il compito di di disattivare la DNAsi tramite la sottrazione degli ioni magnesio \ch{Mg+}
	\item \textbf{\slz II}: contenente \ch{NaOH} e \ch{SDS} (\iupac{Laurilsolfato di sodio}). L'\ch{NaOH} provoca la rottura delle cellule, la denaturazione e precipitazione delle molecole di DNA mentre \ch{SDS} provoca la denaturazione delle proteine tramite la rottura dei legami intermolecolari tra cui legami a idrogeno e interazioni idrofobiche
	\item \textbf{\slz III}: contenente acetato di potassio (\ch{CH3COO- K+}), il quale riporta il lisato cellulare a \pH\ che consente la rinaturazione del DNA plasmidico ma non di quello cromosomico a causa delle dimensioni maggiore  
	\item \textbf{\slz fenolo:cloroformio}: miscela di fenolo saturo e cloroformio in rapporto 1:1. Essendo che è una soluzione molto volatile si aggiunge isoamilico, il quale crea una fase al disopra della soluzione riducendo la possibilità di evaporare. 
	\item \textbf{\slz di etanolo} (\qtylist{100;70}{\percent} v\textbackslash v)
	\item \textbf{\slz di TE}: contenete Tris ed EDTA
\end{itemize}

\subsection{Protocollo}
\subsubsection{Preparazione del campione}
\begin{enumerate}
	\item Prelevare \qty{1.5}{\ml} di coltura batterica sotto la cappa biologica (\autoref{img:01} e \ref{img:02}).          
	\item Centrifuga per 30 secondi la cultura per far si che si creino due fasi: la fase liquida contenente il terreno di coltura (\autoref{img:04}), che verrà eliminata, e la fase solida, ovvero il pellet batterico, dove sono contenuti i nostri batteri (\autoref{img:03})            
\end{enumerate}
\subsubsection{Depurazione del campione}
\begin{enumerate}
	\item Aggiungere \qty{100}{\micro\litre} di Soluzione I al pellet batterico (\autoref{img:05})
	\item Mescolare la soluzione tramite il \foreignlanguage{english}{vortex}
	\item Aggiungere \qty{200}{\micro\litre} di Soluzione II e mescolare capovolgendo la provetta 2 o 3 volte. In questa fase, la soluzione alcalina provoca la lisi delle cellule e la denaturazione e precipitazione del DNA batterico (\autoref{img:06})    
	\item Aggiungere \qty{150}{\micro\litre} di Soluzione III dopo \qtyrange{2}{3}{\min} dalla Soluzione II e mescolare capovolgendo la provetta 2 o 3 volte.    
	\item Centrifuga per \qty{5}{\min} in modo che i residui cellulari e il DNA cromosomico precipitino (\autoref{img:07}). Prelevare la frazione liquida e travasarla in una nuova provetta (\autoref{img:08})
\end{enumerate}
La soluzione ottenuta non contiene DNA plasmidico puro perché contaminato da proteine, oltre a essere stato molto diluito durante le fasi precedenti. Per la purificazione dalle proteine si esegue un trattamento con fenolo-cloroformio.
\subsubsection{Trattamento con fenolo-cloroformio}
\begin{enumerate}
	\item Aggiungere 1 volume (\qty{500}{\micro\litre}) di fenolo-cloroformio per estrarre le proteine (\autoref{img:10}).  
	\item  Centrifuga la soluzione per \qty{3}{\min}, si creeranno due fasi: quella inferiore composta da fenolo-cloroformio e quella superiore con il DNA plasmidico
	\item Recuperare la fase acquosa contenente il DNA plasmidico (\autoref{img:11})
	\item Aggiungere 2 volumi di etanolo \qty{100}{\percent} (\autoref{img:12})
	\item Mettere la soluzione ottenuta in freezer a \qty{-20}{\celsius} per favorire la precipitazione del DNA e poi centrifugare per \qty{5}{\min} (\autoref{img:13}). Si ottiene un precipitato biancastro contenente il DNA plasmidico
	\item Rimuovere la fase alcolica all’interno della cappa chimica (\autoref{img:14})
	\item Lavare il precipitato con etanolo \qty{70}{\percent} per eliminare i sali presenti e centrifugare per \qty{5}{\min}
	\item Rimuovere completamente l’etanolo prima con una micropipetta e poi facendo seccare all’aria per far evaporare eventuali residui
\end{enumerate}

\subsubsection{Conservazione del DNA plasmidico}
\begin{enumerate}
	\item  Per risospendere il DNA plasmidico, aggiungere \qty{50}{\micro\litre} di TE \pH\ 8 per evitare la degradazione del DNA da parte delle DNAsi (\autoref{img:15})
	\item  Conservare la soluzione concentrata di DNA plasmidico a \qty{-20}{\celsius}
\end{enumerate}

\begingroup
\newpage
% \newgeometry{top=1in,bottom=1in,right=1in,left=1in}
% trim = {left bottom right top}

\photo{
	\img{1-02.jpeg}{\\Prelevamento della \\coltura batterica}{01}{trim={15cm 15cm 15cm 25cm},clip}
}{
	\img{1-03.jpeg}{\\Coltura batterica}{02}{trim={15cm 25cm 15cm 15cm},clip}
}{
	\img{1-04.jpeg}{\\Batteri Precipitati}{03}{trim={15cm 25cm 15cm 15cm},clip}
}

\vspace{0.5cm}
\photo{
	\img{1-05.jpeg}{\\Batteri senza terreno}{04}{trim={10cm 20cm 9cm 4cm},clip}   
}{
	\img{1-06.jpeg}{\\Aggiunta Soluzione I}{05}{trim={17cm 25cm 15cm 15cm},clip}
}{
	\img{1-16.jpeg}{\\Aggiunta soluzione II}{06}{trim={15cm 25cm 15cm 15cm},clip}	
}
		
\vspace{0.5cm}
\photo{
	\img{1-07.jpeg}{\\Centrifuga Soluzione III}{07}{trim={15cm 20cm 15cm 20cm},clip}
}{
	\img{1-08.jpeg}{\\DNA e proteine di scarto}{08}{trim={15cm 20cm 15cm 20cm},clip}
}{
	\img{1-09.jpeg}{\\DNA plasmidico in soluzione}{09}{trim={15cm 20cm 15cm 20cm},clip}
}
	
\vspace{0.5cm}
\photo{
	\img{1-10.jpeg}{\\Aggiunta cloroformio}{10}{trim={15cm 10cm 15cm 30cm},clip}
}{
	\img{1-11.jpeg}{\\Recupero fase \\superiore DNA}{11}{trim={15cm 20cm 15cm 20cm},clip}
}{
	\img{1-12.jpeg}{\\Aggiunta di\\ etanolo \qty{100}{\percent}}{12}{trim={15cm 20cm 15cm 20cm},clip}
}
	
\vspace{0.5cm}
\photo{
	\img{1-13.jpeg}{\\Centrifuga dopo\\ aggiunta di etanolo \qty{100}{\percent}}{13}{trim={10cm 15cm 20cm 25cm},clip}	
}{
	\img{1-14.jpeg}{\\Rimozione etanolo \qty{100}{\percent}}{14}{trim={10cm 10cm 25cm 30cm},clip}
}{
	\img{1-15.jpeg}{\\DNA plasmidico in\\ soluzione di TE}{15}{trim={15cm 20cm 15cm 20cm},clip}
}


\endgroup