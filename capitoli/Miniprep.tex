\section{Miniprepazione del DNA plasmidico}

\subsection{Materiali e composti utilizzati}
\paragraph{Strumentazione}
\begin{itemize}[person]
	\item Pipette \foreignlanguage{german}{Eppendorf} da \qtylist{1.5;2.5}{\ml}
	\item Propipette
	\item \foreignlanguage{english}{Vortex}
	\item Centrifuga
	\item Freezer
\end{itemize}

\paragraph{Composti utilizzati}
\begin{itemize}[person]
	\item \textbf{Coltura batterica} di batteri resistenti alla amoxicillina (antibiotico)
	\item \textbf{\slz I}: soluzione contenente EDTA, composto chelante che ha il compito di di disattivare la DNAsi tramite la sottrazione degli ioni magnesio \ch{Mg+}
	\item \textbf{\slz II}: contenente \ch{NaOH} e \ch{SDS} (Laurilsolfato di sodio). L'\ch{NaOH} provoca la rottura delle cellule, la denaturazione e precipitazione delle molecole di DNA mentre \ch{SDS} provoca la denaturazione delle proteine tramite la rottura dei legami intermolecolari tra cui legami a idrogeno e interazioni idrofobiche
	\item \textbf{\slz III}: contenente acetato di potassio (\ch{CH3COO- K+}), il quale riporta il lisato cellulare a \pH\ che consente la rinaturazione del DNA plasmidico ma non di quello cromosomico a causa delle dimensioni maggiore  
	\item \textbf{\slz fenolo:cloroformio}: miscela di fenolo saturo e cloroformio in rapporto 1:1. Essendo che è una soluzione molto volatile si aggiunge isoamilico, il quale crea una fase al disopra della soluzione riducendo la possibilità di evaporare. 
	\item \textbf{\slz di etanolo} (\qtylist{100;70}{\percent} v\textbackslash v)
	\item \textbf{\slz di TE}: contenete
\end{itemize}

\subsection{Protocollo}
\subsubsection{Preparazione del campione}
\begin{enumerate}
	\item Prelevare \qty{1.5}{\ml} di coltura batterica sotto la cappa biologica.          
	\item Centrifuga per 30 secondi la cultura per far si che si creino due fasi: la fase liquida contenente il terreno di coltura, che verrà eliminata, e la fase solida, ovvero il pellet batterico, dove sono contenuti i nostri batteri            
\end{enumerate}
\subsubsection{Depurazione del campione}
\begin{enumerate}
	\item Aggiungere \qty{100}{\micro\litre} di Soluzione I al pellet batterico.
	% TODO: ? contiene RNasi per evitare che la soluzione sia contaminata da RNA  ?         
	\item Mescolare la soluzione tramite il \foreignlanguage{english}{vortex}
	\item Aggiungere \qty{200}{\micro\litre} di Soluzione II e mescolare capovolgendo la provetta 2 o 3 volte. In questa fase, la soluzione alcalina provoca la lisi delle cellule e la denaturazione e precipitazione del DNA batterico    
	\item Aggiungere \qty{150}{\micro\litre} di Soluzione III dopo \qtyrange{2}{3}{\min} dalla Soluzione II e mescolare capovolgendo la provetta 2 o 3 volte.    
	\item Centrifuga per \qty{5}{\min} in modo che i residui cellulari e il DNA cromosomico precipitino. Prelevare la frazione liquida e travasarla in una nuova provetta.
\end{enumerate}
La soluzione ottenuta non contiene DNA plasmidico puro perché contaminato da proteine, oltre a essere stato molto diluito durante le fasi precedenti. Per la purificazione dalle proteine si esegue un trattamento con fenolo-cloroformio.
\subsubsection{Trattamento con fenolo-cloroformio}
\begin{enumerate}
	\item  Aggiungere 1 volume (\qty{500}{\micro\litre}) di fenolo-cloroformio per estrarre le proteine.  
	\item  Centrifuga la soluzione per \qty{3}{\min}, si creeranno due fasi: quella inferiore composta da fenolo-cloroformio e quella superiore con il DNA plasmidico
	\item  Recuperare la fase acquosa contenente il DNA plasmidico
	\item  Aggiungere 2 volumi di etanolo \qty{100}{\percent} (\qty{800}{\micro\litre})
	\item  Mettere la soluzione ottenuta in freezer a \qty{-20}{\celsius} per favorire la precipitazione del DNA e poi centrifugare per \qty{5}{\min}. Si ottiene un precipitato biancastro contenente il DNA plasmidico
	\item  Rimuovere la fase alcolica all’interno della cappa chimica
	\item Lavare il precipitato con etanolo \qty{70}{\percent} per eliminare i sali presenti e centrifugare per \qty{5}{\min}
	\item  Rimuovere completamente l’etanolo prima con una micropipetta e poi facendo seccare all’aria per far evaporare eventuali residui
\end{enumerate}

\subsubsection{Conservzione del DNA plasmidico}
\begin{enumerate}
	\item  Risospendere il DNA plasmidico in \qty{50}{\micro\litre} di TE (Tris + EDTA) \pH\ 8 per evitare la degradazione del DNA da parte delle DNAsi
	\item  Conservare la soluzione concentrata di DNA plasmidico a \qty{-20}{\celsius}
\end{enumerate}