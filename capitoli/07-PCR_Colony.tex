\section{PCR su colonia -- \foreignlanguage{english}{Colony PCR}}

\subsection{Materiali e composti utilizzati}
\paragraph{Strumentazione}
\begin{itemize}
	\itemb[Termociclatore:] descritto \autoref{it:termociclatore}. In questo caso il programma utilizzato è il seguente:
	\begin{itemize}[squareItem]
		\itemb[Denaturazione iniziale:] \qty{3}{\min} a \qty{95}{\celsius}
		\itemb[Amplificazione] per 28 cicli, ogni ciclo si divide in:
		\begin{itemize}[squareItem]
			\itemb[Denaturazione]: \qty{30}{\sec} a \qty{95}{\celsius}
			\itemb[Ibridazione]: \qty{30}{\sec} a \qty{55}{\celsius}
			\itemb[Estensione]: \qty{60}{\sec} a \qty{72}{\celsius}
		\end{itemize}
		\itemb[Estensione finale:] \qty{1}{\min} a \qty{72}{\celsius}
	\end{itemize}
\end{itemize}

\paragraph{Composti utilizzati}
\begin{itemize}
	\itemb[PCR mix]: è una miscela per PCR ottimizzata e pronta all'uso contenente la \enzima{TAQ polimerasi}, il buffer per PCR, \ch{MgCl2} e \glsentryshortpl{dntp}.
	La Mix contiene tutti i componenti per la PCR, a eccezione dei primers e del DNA stampo.
\end{itemize}


\subsection{Protocollo}
\begin{Note}
	Durante l'esperienza è stata utilizzata la piastre preparata nell'\hyperref[sec:6]{esperienza 6}, la quale presentava colonie blu e colonie bianche.

	\vspace*{.2cm}
	La diversa colorazione delle colonie dipende dalla presenza o meno dell'inserto nella cellula. L'inserto impedisce la trascrizione della \iupac{\b-galattossidasi}, la quale non metabolizzando l’\gls{X-gal} non produce il \iupac{5-bromo-4-cloro-3-idrossindolo} che è il responsabile della colorazione blu delle cellule.
\end{Note}

\noindent Procedura operativa:
\begin{enumerate}
	\item Prendere due provette da PCR e inserire \qty{20}{\micro\litre} di acqua sterile in ognuna
	\item Con la punta di una pipetta toccare una colonia bianca e dissolverla nella provetta contenente acqua. Ripetere la stessa operazione per la colonia blu.
	\item Prendere due nuove provette da PCR e inserirci \qty{15}{\micro\litre} di PCR mix (verde) in ciascuna
	\item Aggiungere \qty{5}{\micro\litre} di ciascuna colonia al PCR mix
	\item Mescolare vortexando le due soluzioni
	\item Inserire le provette nel termociclatore e far partire la PCR
	\item Una volta finita la PCR, inserire i campioni sul gel di corsa (precedentemente preparato come descritto in \ref{sssec:agarosio})
	\item Alla fine, vedere il gel allo spettrofotometro UV
\end{enumerate}

\noindent I risultati attesi prevedono la presenza di due bande:
\begin{itemize}
	\item Una molto sottile e bassa (poche basi) che rappresenta il plasmide vuoto delle colonie blu.
	\item Una più spessa e alta (tante basi perché l’inserto è grande) che rappresenta il plasmide con l’inserto delle colonie bianche.
\end{itemize}