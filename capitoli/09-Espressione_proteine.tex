\section{Espressione eterologa di proteine e loro purificazione}\label{sec:espProteine}

\subsection{Materiali e composti utilizzati}
% \paragraph{Strumentazione}


\paragraph{Composti utilizzati}
% \newcommand{\squarebullet}{\raisebox{.45ex}{\rule{.6ex}{.6ex}}}
\begin{itemize}
	\begin{multicols}{2}
		\itemb[Tampone di Lisi]:
		\begin{itemize}[squareItem]
			\item \qty{50}{\milli\Molar} \gls{tris} \pH\ 8
			\item \qty{25}{\percent} saccarosio
			\item \qty{1}{\milli\Molar} \gls{edta}
		\end{itemize}
		\itemb[Tampone Detergente]:
		\begin{itemize}[squareItem]
			\item \qty{200}{\milli\Molar} \ch{NaCl}
			\item \qty{1}{\percent} \iupac{acido deossicolico}
			\item \qty{1}{\percent} NONIDET P-40
			\item \qty{20}{\milli\Molar} \gls{tris} \pH\ \num{7.5}
			\item \qty{2}{\milli\Molar} \gls{edta}
			\item \qty{10}{\milli\Molar} \iupac{\b-mercaptoetanolo}
		\end{itemize}
		\itemb[Tampone Triton]:
		\begin{itemize}[squareItem]
			\item \qty{0.5}{\percent} m/V di Triton X-100
			\item \qty{20}{\milli\Molar} \gls{tris-hcl} \pH\ \num{7.5}
			\item \qty{1}{\milli\Molar} di \iupac{\b-mercaptoetanolo}
		\end{itemize}
		\itemb[Tampone di Ricostituzione]:
		\begin{itemize}[squareItem]
			\item \qty{50}{\milli\Molar} \gls{hepes} \pH\ 8,
			\item \qty{12.5}{\percent} saccarosio,
			\item \qty{2}{\percent} \gls{lds},
			\item \qty{5}{\milli\Molar} \iupac{acido 6-aminocaproico}
			\item \qty{1}{\milli\Molar} \iupac{benzamidina}
		\end{itemize}
	\end{multicols}
\end{itemize}

\subsection{Protocollo}

\subsubsection{Purificazione LHC}
\begin{enumerate}
	\item Pesare una provetta Falcon da \qty{150}{\ml} e annotare il peso.
	      \begin{Note}
		      Noi abbiamo preparato tre provette Falcon da \qty{50}{\ml} (\autoref{img:9-01}) e le abbiamo usate come soluzioni a se stanti fino al \autoref{it:pellet}, successivamente le abbiamo unite in una sola prima di aggiungere il lisozima.
	      \end{Note}
	\item Prelevare \qty{50}{\ml} di coltura batterica e inserirli nella Falcon.
	\item Centrifugare a \qty{4350}{\giri} per \qty{5}{\min}
	\item Eliminare il surnatante e ripesare la provetta contente il pellet (\autoref{img:9-02}).
	\item \label{it:pellet} Il pellet pesato viene risospeso in \qty{0.8}{\ml\per\g} di tampone di lisi preparato precedentemente (\autoref{img:9-03}).
	\item Aggiungere una spatola di lisozima e lasciare a temperatura ambiente per \qtyrange{20}{25}{\min} (\autoref{img:9-04})
	      \begin{Informazione}
		      A seconda della quantità di lisozima aggiunta dipenderà la velocità di reazione.
	      \end{Informazione}
	\item Aggiungere \qty{10}{\micro\litre} di \enzima{DNAsi} e \qty{10}{\milli\Molar} \ch{MgCl2} (\autoref{img:9-05})
	      \begin{myBox}
		      Abbiamo calcolato la quantità di prelievo di \ch{MgCl2} tramite la formula di diluizione:
		      \begin{gather*}
			      V_i \cdot C_i = V_f \cdot V_i \\
			      x \cdot \qty{1}{\Molar}= \qty{0.56}{\ml} \cdot \qty{0.01}{\Molar}\\
			      \ch{MgCl2} \rightarrow \frac{\qty{0.56}{\ml} \cdot \qty{0.01}{\Molar}}{\qty{1}{\Molar}} = \qty{5.6}{\micro\litre}
		      \end{gather*}
	      \end{myBox}
	\item Incubare per \qty{10}{\min} a temperatura ambiente.
	\item Aggiungere \qty{1}{\ml\per\g} di tampone detergente (\qty{0.7}{\ml}).
	\item Centrifugare per \qty{10}{\min} a \qty{12000}{\giri} in modo da precipitare i corpi di inclusione.
	\item Lavare il pellet con tampone Triton \qty{1}{\ml\per\g} (\autoref{img:9-06}).
	      \begin{Note}
		      Il lavaggio viene effettuato per rimuovere i contaminanti batterici e isolare i corpi di inclusione.
	      \end{Note}
	\item Centrifugare per \qty{10}{\min} a \qty{12000}{\giri} in modo da precipitare i corpi di inclusione.
	\item Lavare i corpi di inclusione in \qty{500}{\micro\litre} di \ch{H2O} per togliere eventuali residui di Triton.
	\item Centrifugare per 10 minuti a \qty{12000}{\giri} in modo da precipitare i corpi di inclusione.
	\item Solubilizzare i corpi di inclusione nel tampone di ricostituzione (\autoref{img:9-08}). 
	\begin{Note}
		Al fine di evitare la formazione di schiuma, risospendere il pellet in \qty{200}{\micro\litre} di acqua e aggiungere \qty{200}{\micro\litre} di tampone di ricostituzione 2x. Il tampone di ricostituzione contiene un detergente talmente forte da scogliere i corpi di inclusione.
	\end{Note}
\end{enumerate}

\subsection{Purificazione GFP}
\begin{enumerate}
	\item Pesare un tubo Falcon da \qty{50}{\ml} e annotare il peso
	\item Prelevare \qty{50}{\ml} di coltura batterica e inserirli nella Falcon (\autoref{img:9-01})
	\item Centrifugare a \qty{4350}{\giri} per \qty{5}{\min}
	\item Recuperare il terreno e ripesare la provetta (\autoref{img:9-02})
	\item Il pellet  pesato viene risospeso in \qty{0.8}{\ml\per\g} di tampone di lisi (\autoref{img:9-03})
	\item Aggiungere una spatola di lisozima (stock \qty{50}{\mg\per\ml}) (\autoref{img:9-04})
	\item Far riposare in ghiaccio per \qty{20}{\min}
	\item Aggiungere \qty{10}{\micro\litre} di \enzima{DNAsi} (stock \qty{20}{\mg\per\ml})
	\item Aggiungere \qty{10}{\milli\Molar} \ch{MgCl2} (stock \qty{1}{\Molar})
	\item Incubare per \qty{10}{\min} a temperatura ambiente
	\item Centrifugare per  \qty{10}{\min} a \qty{12000}{\giri}
	\item Recuperare il surnatante (\autoref{img:9-07})
\end{enumerate}



\begin{table}[H]
	\begin{tblr}{
		colspec={
		X[c]X[2,c]X[2,c]X[2,c]X[2,c]
		},
		cell{2}{1} = {r=3}{c},
		vline{2-5} = {1}{-}{},
		hline{2,5} = {1}{-}{},
		hline{2,5} = {2}{-}{},
			}
		\toprule
		    & Tara Falcon       & Coltura           & Pellet                                 & Tampone di Lisi                                           \\
		LHC & 12.7675 \unit{\g} & 13.0227 \unit{\g} & \fpMathSub{13.0227}{12.7675} \unit{\g} & \fpMathMult{\fpMathSub{13.0227}{12.7675}}{0.8} \unit{\ml} \\
		    & 12.5282 \unit{\g} & 12.7336 \unit{\g} & \fpMathSub{12.7336}{12.5282} \unit{\g} & \fpMathMult{\fpMathSub{12.7336}{12.5282}}{0.8} \unit{\ml} \\
		    & 12.5312 \unit{\g} & 12.7628 \unit{\g} & \fpMathSub{12.7628}{12.5312} \unit{\g} & \fpMathMult{\fpMathSub{12.7628}{12.5312}}{0.8} \unit{\ml} \\
		GFP & 12.69 \unit{\g}   & 12.9907 \unit{\g} & \fpMathSub{12.9907}{12.69} \unit{\g}   & \fpMathMult{\fpMathSub{12.9907}{12.69}}{0.8} \unit{\ml}   \\
		\bottomrule
	\end{tblr}
	\caption{Dati sperimentali}\label{tab:pesiRel9}
\end{table}



\begingroup
\newpage
\vspace*{-1.2cm}
\photo{
	\img{9-01}{\\Soluzioni di LHC e GFP}{9-01}{}
}{
	\img{9-02}{\\Pellet LHC e GFP\\\phantom{}}{9-02}{}
}{
	\img{9-03}{\\Aggiunta di buffer di lisi}{9-03}{}
}

% \vspace{0.1cm}
\photo{
	\img{9-04}{\\Aggiunta di lisozimi\\\phantom{}}{9-04}{}
}{
	\img{9-05}{\\LHC con aggiunta di \ch{MgCl2} e \enzima{DNAsi}}{9-05}{trim={6cm 9cm 6cm 9cm}, clip}
}{
	\img{9-06}{\\LHC -- primo lavaggio con Triton}{9-06}{trim={6cm 9cm 6cm 9cm}, clip}
}

% \vspace{-0.05cm}
\photo{
	\img{9-07}{\\GFP surnatande (a sinistra) e precipitato\\\phantom{}}{9-07}{trim={6cm 9cm 6cm 9cm}, clip}
}{
	\img{9-08}{\\Risospensione nel tampone di ricostituzione}{9-08}{trim={8cm 25cm 8cm 9cm}, clip}
}{
}
\endgroup